\documentclass[12pt,pdftex,a4paper]{article}
\usepackage[ngerman]{babel}
\usepackage{amsmath}
\usepackage{amssymb}
\usepackage{bbm}
\newcommand{\bbN}{\mathbbm{N}}
\newcommand{\bbR}{\mathbbm{R}}
\newcommand{\bbZ}{\mathbbm{Z}}
\newcommand{\bbI}{\mathbbm{I}}
\usepackage[pdftex]{graphicx}
\usepackage{listings}
\lstset{language=Python,basicstyle=\footnotesize}
\begin{document}
\title{Projekt-INF Anforderungen}
\author{Patrick Spaney, Kai Ziegler, Jonas Kittelberger, Raphael Br"osamle}
\date{Institut f"ur Formale Methoden der Informatik \\ Universit"at Stuttgart}
\pagenumbering{gobble}
\maketitle
%%%%%%%%%%%%%%%%%%%%%%%%%%%%%%%%%%%%%%%%%%%%%%%%
\[\]
Im Rahmen des Projekt-INF's sollen verschiedene Varianten von in-place Mergesort Algorithmen implementiert und mit bestehenden in-place Mergesort Algorithmen verglichen werden. \\
Dabei sollen folgendes implementiert werden:
\begin{itemize}
\item Algorithmus von Chen
\item Algorithmus von Reinhardt
\item Eigene Optimierungsans"atze
\end{itemize}
F"ur die Implementierung wird die Programmiersprache C++ verwendet. Es wird ein gemeinsames Interface zum Ausf"uhren der Sortieralgorithmen verwendet. Um die Funktionalit"at und die Performanz der Implementierungen festzustellen werden Tests geschrieben und die Anzahl der Vergleiche, so wie die Anzahl der Vertauschungen ausgegeben.\\
Die Ergebnisse sollen in einem, bis zu 10 Seiten umfassenden, Dokument festgehalten werden. Es soll in der Englischen Sprache verfasst werden.  \\
Das Dokument beinhaltet: 
\begin{itemize}
\item Literaturrecherche "uber in-place Mergesort Algorithmen
\item Eine grobe Beschreibung der Implementierungen
\item Vergleiche der Implementierungen mit bestehenden in-place Mergesort Algorithmen
\end{itemize}

%%%%%%%%%%%%%%%%%%%%%%%%%%%%%%%%%%%%%%%%%%%%%%%%
\end{document}

