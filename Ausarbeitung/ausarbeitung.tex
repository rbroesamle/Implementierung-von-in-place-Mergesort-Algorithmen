\documentclass[12pt,pdftex,a4paper]{article}
\usepackage[ngerman]{babel}
\usepackage[utf8]{inputenc}
\usepackage{url}
\setlength {\marginparwidth }{2cm}
\usepackage{todonotes}
\include{sort}
%\usepackage[pdftex]{graphicx}
\usepackage{listings}
\lstset{language=Python,basicstyle=\footnotesize}
\setlength{\parindent}{0em}
\begin{document}
\title{Projekt-INF:\\
Implementierung von In-place Mergesort Algorithmen}
\author{Patrick Spaney, Kai Ziegler, Jonas Kittelberger, \\ Raphael Brösamle}
\date{Institut für Formale Methoden der Informatik \\ Universität Stuttgart\\
\normalsize Betreuer: Dr. Armin Weiß\\
Prüfer: Prof. Dr. Volker Diekert}
\pagenumbering{gobble}
\maketitle
%%%%%%%%%%%%%%%%%%%%%%%%%%%%%%%%%%%%%%%%%%%%%%%%
%%%%%%%%%%%%%%%%%%%%%%%%%%%%%%%%%%%%%%%%%%%%%%%%
\section*{Literaturrecherche \"uber in-place Mergesort Algorithmen}

Mergesort ist ein Sortieralgorithmus, der nach dem Prinzip ''Teile - und - Herrsche'' arbeitet. Beim ''Teile'' - Schritt wird die zu
sortierende Liste in zwei kleinere Listen zerlegt. Beim ''Herrsche'' - Schritt werden zwei kleinere, bereits sortierte Listen,
mittels paarweisen Vergleichen der Elemente zu einer sortierten Liste zusammengesetzt.
Der gewöhnliche Algorithmus erreicht dabei die asymptotisch optimale Worst-Case-Laufzeit von $O(n \log n)$ und
lässt sich mit externem Speicher auf verschiedenen Kernen gut parallelisieren und weiter beschleunigen.
Will man die Liste ohne zusätzlichen Speicher (in-place) sortieren, hat neben der Anzahl der Vergleiche
auch die Anzahl der Vertauschungen innerhalb der Liste einen sehr großen Einfluss auf die Laufzeit.
Daher benötigt der ''klassische'' Algorithmus diverse Änderungen und Optimierungen, um ohne
zusätzlichen Speicher und trotzdem weiterhin effizient zu sortieren. \\
Es gibt verschiedene Ans\"atze die versuchen dies zu erreichen. Viele versuchen dabei einen normalen Mergesort mit einem in-place Mergesort zu kombinieren und somit eine schnellere Sortierung zu erreichen. Diese Algorithmen sind jedoch streng genommen keine in-place Mergesort Algorithmen, da der normale Mergesort logarithmischen Platz f\"ur seinen internen Stack ben\"otigt. Dieser quasi-linearer Extra-Speicher ist aber in den mei{\ss}ten F\"allen ausreichend. \\
Im sp\"ateren Verlauf dieser Ausarbeitung werden wir vor allem auf den Algorithmus von Chen\cite{Chen06} und den Algorithmus von Reinhardt\cite{Reinhardt92} genauer eingehen. 

\section*{Beschreibung der Implementierungen}

\section*{Detaillierte Vergleiche der Implementierungen mit bestehenden in-place Mergesort Algorithmen}

\bibliographystyle{plain}	
%	\section*{Literaturhinweise}
\bibliography{sort}

\end{document}

